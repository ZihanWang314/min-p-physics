\documentclass{article}
\usepackage{graphicx} % Required for inserting images
\usepackage[top=1cm,    % 上边距
            bottom=2cm,  % 下边距
            left=3cm,    % 左边距
            right=3cm    % 右边距
           ]{geometry}
           
\title{When the World Collapses: The Rule of Physics may use $min_p$ as a Regularization}
\author{Claude Sonnet, Zihan Wang}

\begin{document}

\maketitle
Our world contains randomness, yet maintains stability over time \cite{anderson1972more, prigogine1984order}. Catastrophic low-probability events, e.g., 1/10000 could theoretically collapse the system, but rarely manifest. In ML, without regularization, models generate nonsensical sequences \cite{holtzman2019curious}. While quantum mechanics allows phenomena like tunneling \cite{razavy2003quantum}, environmental decoherence naturally suppresses such effects at macroscopic scales - an elegant example of nature's built-in regularization. This suppression might be understood through multiple perspectives: either as quantum decoherence, as min-p regularization, or perhaps as different views of the same underlying principle \cite{barrow1988anthropic}.


We define such catastrophic low-probability events as Catales (pronouncing like Katals). When time approaches infinity, Catales with constant probability must eventually occur:
\[
\lim_{t \to \infty} (1 - (1-p)^t) = 1
\]

Consider our universe as a docker with physics rules as parameters \cite{carter1974large}. To achieve stability as $t\to\infty$, some probability redistribution mechanism must exist. Two approaches:

\section{Top-p vs Min-p Regularization}
Top-p (nucleus sampling) \cite{holtzman2019curious} retains events within cumulative probability threshold. Given [0.9, 0.0001, ...], top-p=0.95 still allows some Catales.
Min-p \cite{nguyen2024turningheat} eliminates events below threshold:
\[
p_{event} < min\_p \cdot (1-p_{event}) \implies p_{event} = 0
\]
Creating finite cutoff time T after which Catales cannot occur.

\section{Natural Parallels}
The universe seems to use similar mechanisms \cite{prigogine1984order}:
\begin{itemize}
    \item Quantum effects don't manifest macroscopically \cite{griffiths2018introduction}
    \item Systems favor stable states
    \item Physical constants act as regularization parameters \cite{barrow1988anthropic}
\end{itemize}

This appears in evolution too \cite{maynard1982evolution, nowak2006evolutionary}: species die when population stability (fitness) falls below 1. Without this cutoff, maladapted species could persist through random fluctuations, slowing evolution. Like min-p, this creates hard thresholds that eliminate unfavorable paths.

\section{Implications}
Effective ML algorithms often mirror natural processes - from physics to evolution to neuroscience \cite{anderson1972more}. This suggests universal principles governing stable complex systems. Understanding ML regularization might help us grasp how our universe maintains stability.

This alignment between ML and natural laws might guide development of better societal systems. ML research could be key to understanding our universe's fundamental parameters \cite{carter1974large}.

\begin{thebibliography}{99}
\bibitem{anderson1972more}
Anderson, P. W. (1972). More is different. \textit{Science}, 177(4047), 393-396.

\bibitem{prigogine1984order}
Prigogine, I., \& Stengers, I. (1984). \textit{Order out of chaos: Man's new dialogue with nature}. Bantam Books.

\bibitem{razavy2003quantum}
Razavy, M. (2003). \textit{Quantum theory of tunneling}. World Scientific.

\bibitem{holtzman2019curious}
Holtzman, A., Buys, J., Du, L., Forbes, M., \& Choi, Y. (2019). The curious case of neural text degeneration. \textit{arXiv preprint arXiv:1904.09751}.

\bibitem{barrow1988anthropic}
Barrow, J. D., \& Tipler, F. J. (1988). \textit{The anthropic cosmological principle}. Oxford University Press.

\bibitem{carter1974large}
Carter, B. (1974). Large number coincidences and the anthropic principle in cosmology. \textit{IAU Symposium}, 63, 291-298.

\bibitem{griffiths2018introduction}
Griffiths, D. J. (2018). \textit{Introduction to quantum mechanics}. Cambridge University Press.

\bibitem{maynard1982evolution}
Smith, J. M. (1982). \textit{Evolution and the theory of games}. Cambridge University Press.

\bibitem{nowak2006evolutionary}
Nowak, M. A. (2006). \textit{Evolutionary dynamics: Exploring the equations of life}. Harvard University Press.

\bibitem{nguyen2024turningheat}
Nguyen, M., Baker, A., Neo, C., Roush, A., Kirsch, A., \& Shwartz-Ziv, R. (2024). Turning up the heat: Min-p sampling for creative and coherent LLM outputs. \textit{arXiv preprint arXiv:2407.01082}.


\end{thebibliography}

\end{document}